\section{JDBC}

\subsection{Type Mapping}
\begin{tabular}{|l|l|}
    \hline
    \textbf{Datenbanktyp} & \textbf{Java Datentyp} \\
    \hline
    CHAR, VARCHAR, LONGVARCHAR & String \\
    \hline
    BIT, BOOLEAN &  boolean \\
    \hline
    INTEGER & int \\
    \hline
    BIGINT & long \\
    \hline
    REAL & float \\
    \hline
    FLOAT, DOUBLE & double \\
    \hline
\end{tabular}

\subsection{Transaction Levels}
\begin{tabular}{|l|c|l|}
    \hline
    \textbf{Level} & \textbf{Wert} & \textbf{Beschreibung} \\
    \hline
    TRANSCATION\_NONE & 0 & Es werden keine Sperren in der DB gesetzt \\
    \hline
    TRANSACTION\_READ\_UNCOMMITTED & 1 & Lesende Transaktionen verursachen keine Sperren. \\
    \hline
    TRANSACTION\_READ\_COMMITTED & 2 & Lesende Transaktionen verursachen Sperren \\
    \hline
    TRANSACTION\_SERIALIZABLE & 3 & \specialcell{Transaktionen werden geblockt und hintereinander\\ ausgeführt.} \\
    \hline
\end{tabular}

\subsection{Prepared Statement}

\begin{lstlisting}[style=Java]
    PreparedStatement updateOrt;
    
    String updateString = "UPDATE Dozent SET ort = ?";
    updateOrt = connection.prepareStatement(updateString);
    
    updateOrt.setString("1, Zuerich");
\end{lstlisting}

\subsection{Metadaten}
Metadaten geben den JDBC Benutzer Informationen über die Datenbank die im DBMS Schema gespeichert sind.
\begin{lstlisting}[style=Java]
    DatabaseMetaData dbmd = connection.getMetaData();
    String URL = dbmd.getURL;
\end{lstlisting}

\subsection{Code Beispiel}
\lstinputlisting[style=Java]{src/jdbc.java}


