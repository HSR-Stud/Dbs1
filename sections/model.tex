\section{Datenmodellierung}
\subsection{ER-Modell}
TODO: Bild (Datenmodellierung.pdf - S.15) \\
\subsection{UML-Modell}
TODO: Aggregation / Komposition / Assoziative Klasse // Generalisierung \\
TODO: Bild (Datenmodellierung.pdf - S.28) \\
\subsection{Relationales Modell}
\begin{description}
    \item[Entitität] Individuelles Element (Objekt) des betrachteten Systems
    \item[Relation] beschreibt eine Entitätsmenge
    \item[Tupel] repräsentiert eine Entität
\end{description}
TODO: Darstellung durch Tabellen?
\subsubsection{Schlüssel}
\begin{description}
    \item[Schlüssel] Attribute oder eine Kombination von Attributen, die ein Tupel eindeutig identifizieren.
    \item[Primärschlüssel] Der aus den möglichen Schlüsseln (= Schlüsselkandidaten) ausgewählte identifizierende Schlüssel. Wird für Fremdschlüsselbeziehungen verwendet.
    \item[Schlüsselkandidat] Attribut oder Kombination davon, das/die ein Tupel identifiziert.
    \item[Surrogatschlüssel] Künstlicher Schlüssel. Wird häufig eingeführt, wenn keiner der Schlüsselkandidaten obige Eigenschaften erfüllt.
\end{description}
\subsubsection{NULL}
TODO: Whatever\ldots
\subsection{Referentielle Integrität}
TODO: Ken plan\ldots \\

TODO: Begriffe (Datemodellirung.pdf S.50) \\