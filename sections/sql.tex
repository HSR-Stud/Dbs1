\section{SQL}
    \begin{multicols}{2}
    \subsection{Data Definition Language (DDL)}
        \subsubsection{Basisdatentypen}
            \begin{description}
            \setlength{\itemsep}{0pt}
                \item[BOOLEAN] Boolscher Datentyp
                \item[SMALLINT] Ganzzahl (2 Byte)
                \item[INT / INTEGER] Ganzzahl (4 Byte)
                \item[BIGINT] Ganzzahl (8 Byte)
                \item[REAL, FLOAT, DOUBLE] Fliesskomma-Zahl (8 Byte)
                \item[NUMERIC, DECIMAL (beide: precision, scale)] Festkommazahl
                \item[CHAR(size), CHARACTER(size)] String mit fixer Länge
                \item[VARCHAR(size)] String mit variabler Länge
                \item[DATE] Jahr, Monat, Tag
                \item[TIME] Stunde, Minute, Sekunde
                \item[INTERVAL] Zeitintervall
                \item[DATETIME] DATE + TIME
                \item[BINARY, VARBINARY, LONGBINARY] Binäre Datentypen
                \item[CLOB, BLOB] Grosse Text- / Binärdaten
            \end{description}
            \paragraph{Spezelles bei PostgreSQL}
                \begin{itemize}
                \setlength{\itemsep}{0pt}  
                  \item kein FLOAT
                  \item TEXT für Zeichenketten bel. Länge
                  \item kein DATETIME, siehe TIMESTAMP
                \end{itemize}
        \subsubsection{Create Table}
            \lstinputlisting[style=SQL]{src/createtable.sql}
        \subsubsection{Alter Table}
            \lstinputlisting[style=SQL]{src/altertable.sql}
        \subsubsection{Drop Table}
            \lstinputlisting[style=SQL]{src/droptable.sql}
        \subsubsection{Create Index}
            \lstinputlisting[style=SQL]{src/createindex.sql}
        \subsubsection{Drop Index}
            \lstinputlisting[style=SQL]{src/dropindex.sql}
    \subsection{Data Manipulation Language (DML)}
        \subsubsection{Insert}
            \lstinputlisting[style=SQL]{src/insert.sql}
        \subsubsection{Select}
            \lstinputlisting[style=SQL]{src/select.sql}
        \subsubsection{Update}
            \lstinputlisting[style=SQL]{src/update.sql}
        \subsubsection{Delete}
            \lstinputlisting[style=SQL]{src/delete.sql}
        \subsubsection{CTE - Common Table Expressions}
            \lstinputlisting[style=SQL]{src/cte.sql}
    % TODO: Window Fönktschens \\
        \subsubsection{Create View}
            \lstinputlisting[style=SQL]{src/createview.sql}
        \subsubsection{Drop View}
            \lstinputlisting[style=SQL]{src/dropview.sql}
        \subsubsection{Enumerate}
            \lstinputlisting[style=SQL]{src/enum.sql}
        \subsubsection{Domain}
            \lstinputlisting[style=SQL]{src/domain.sql}
    \end{multicols}