\section{SQL}
\subsection{Data Definition Language (DDL)}
\subsubsection{Basisdatentypen}
\begin{description}
    \item[BOOLEAN] Boolscher Datentyp
    \item[SMALLINT] Ganzzahl (2 Byte)
    \item[INT / INTEGER] Ganzzahl (4 Byte)
    \item[BIGINT] Ganzzahl (8 Byte)
    \item[REAL, FLOAT, DOUBLE] Fliesskomma-Zahl (8 Byte)
    \item[NUMERIC (precision, scale), DECIMAL(precision, scale)] Festkommazahl
    \item[CHAR(size), CHARACTER(size)] String mit fixer Länge
    \item[VARCHAR(size)] String mit variabler Länge
    \item[DATE] Jahr, Monat, Tag
    \item[TIME] Stunde, Minute, Sekunde
    \item[INTERVAL] Zeitintervall
    \item[DATETIME] DATE + TIME
    \item[BINARY, VARBINARY, LONGBINARY] Binäre Datentypen
    \item[CLOB, BLOB] Grosse Text- / Binärdaten
\end{description}
Spezelles bei PostgreSQL:
\begin{itemize}
  \item kein FLOAT
  \item TEXT für Zeichenketten bel. Länge
  \item kein DATETIME, siehe TIMESTAMP
\end{itemize}

TODO: ALTER TABLE; CREATE TABLE, DROP TABLE, CREATE INDEX, DROP INDEX \\

\subsection{Data Manipulation Language (DML)}

TODO: INSERT, SELECT, UPDATE, DELETE \\
TODO: Unterbafragen (NOT) IN, ANY, EXIST, ALL \\
TODO: Mengenoperatoren UNION, MINUS; INTERSECT \\
TODO: Window Fönktschens \\
TODO: CTE \\
TODO: CREATE VIEW, DROP VIEW